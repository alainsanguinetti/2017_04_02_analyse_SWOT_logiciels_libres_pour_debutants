\documentclass[xcolor=svgnames]{beamer}

\usepackage[utf8]    {inputenc}
\usepackage[T1]      {fontenc}
\usepackage[french] {babel}

\usepackage{amsmath,amsfonts,graphicx}
\usepackage{beamerleanprogress}


\title
  [\hspace{2em}]
  {Analyse SWOT des solutions libres pour le marché des débutants informatiques}

\author
  [A.S.]
  {Alain Sanguinetti}

\date
  {Dimanche 2 avril 2017}
  
\institute
  {}


\begin{document}

\maketitle

\section
    {Préparatifs}
    
    % Moi
    \begin{frame}{Présentation}
    
        Alain Sanguinetti, \pause
        utilisateur de solutions libres investi, \pause
        en projet de création d’entreprise.
        
    \end{frame}
    
    % Pourquoi cette présentation
    \begin{frame}{Pourquoi cette présentation ??}
    
        Pour avoir une vision d’ensemble de l’horizon.        
        
    \end{frame}
    
    % L'esprit de cette présentation
    \begin{frame}{L’esprit de cette présentation}
    
        C’est une proposition. 
        
    \end{frame}
    
    % La route devant nous
    \begin{frame}{Chemin}
    
        \tableofcontents
    
    \end{frame}
    
    
    % Show the content at the beginning of each section
    \AtBeginSection[]
      {
         \begin{frame}<beamer>
         \frametitle{Chemin}
         \tableofcontents[currentsection]
         \end{frame}
      }
    

\section{Besoins}

    % Messagerie
    \begin{frame}{Messagerie}
        Le premier besoin. \pause
        
        Plusieurs formes : 
        
        \begin{itemize}
            \item courriels
            \item appels vidéos
            \item réseaux sociaux
        \end{itemize} \pause
        
        Une subtilité est de gérer les contacts.
        
    \end{frame}


    % Internet
    \begin{frame}{Navigation sur internet}
    
        Un besoin et une obligation. \pause
        
        Nécessite d’interagir avec des interfaces très hétéroclites.
        
    \end{frame}
    
    
    % Bureautique
    \begin{frame}{Bureautique}
        
        Plusieurs niveaux :
        \begin{enumerate}
            \item Ouvrir des documents reçus.
            \item Remplir des documents.
            \item Créer des documents.
        \end{enumerate} \pause
        
        Attente de pouvoir imprimer et scanner.
        
    \end{frame}
    
    
    % Photos
    \begin{frame}{Les photos}
        
        Principalement du stockage \pause mais nécessité de récupérer les images.
        
    \end{frame}
    
    
    % Jeux
    \begin{frame}{Jeux}
    
        Faire fonctionner les jeux de la grande distribution \pause
        et les jeux Flash
        
    \end{frame}
    
    
    % Besoins transverses
    \begin{frame}{Avoir confiance}
    
        Sentiment de maîtrise \pause
        
        \begin{itemize}
            \item Fiabilité
            \item Répétabilité
            \item Compréhension / transparence
        \end{itemize}
        
    \end{frame}
    
    \begin{frame}{Avoir confiance (bis)}
    
        Se sentir en sécurité \pause et comprendre les dangers
        
    \end{frame}
    

\section{Clientèle}

    % Vieux
    \begin{frame}{Sages}
    
        En 2020, un français sur 3 aura plus de 65 ans.
        
    \end{frame}
    
    % Debutants et enfants
    \begin{frame}{Nouveaux arrivants}
        
        30\% d’internautes en Afrique. \pause
        
        
        L’informatique à l’école.
        
    \end{frame}
    
    % Enfants
    
    % Total
    \begin{frame}{Au total}
        
        Beaucoup de monde.
        
    \end{frame}
    
    
\section{Concurrents}

    % Microsoft
    \begin{frame}{Microsoft}
        
        Windows \pause, l'option par défaut.
        
    \end{frame}
    
    % Apple
    \begin{frame}{Apple}
    
        MacOS \pause, le choix des plus riches \pause, perçu comme plus simple
        
    \end{frame}
    
    % Google
    \begin{frame}{Google}
    
        ChromeOS \pause, en progression \pause, construit pour Chrome
        
    \end{frame}
    
    % Ordissimo
    \begin{frame}{Ordissimo}
        
        Notoriété auprès des personnes agées. \pause Basé sur Linux.
        
    \end{frame}
    
    % Autre
    \begin{frame}{Globalement}
    
        Pas d'offre spécifique destinée aux débutants \pause mais plutôt des qualité intrinsèques d'un système ou d'un autre.
        
    \end{frame}
    
    
\section{Solutions libres}

    % HandyLinux
    \begin{frame}{DFLinux}
    
        La suite du projet "HandyLinux" \pause
        
        Conçue pour les débutants.   
        
    \end{frame}
    
    % Firefox
    \begin{frame}{Firefox}
    
        Moins répandu \pause mais toujours à jour.
        
    \end{frame}
    
    % Thunderbird
    \begin{frame}{Thunderbird}
    
        Stable
        
    \end{frame}
    
    % LibreOffice
    \begin{frame}{Libreoffice}
    
        Stable \pause, trop stable ?
        
    \end{frame}
    
    % The Gimp
    \begin{frame}{Photos}
    
        Shotwell, Darktable, Gimp, etc \pause mais pas photofiltre :O
        
    \end{frame}
    
    
\section{Forces}

    % Solutions disponibles gratuitement
    \begin{frame}{Prix}
    
        Prix d’achat très avantageux
        
    \end{frame}
    
    % Standardisation / compatibilité
    \begin{frame}{Standardisation}
    
        Dans l’esprit des logiciels libres
        
    \end{frame}
    
    % Légèreté / Frugalité
    \begin{frame}{Légèreté}
        
        Système total < 8 Go
        
    \end{frame}
    
    % Fiabilité / continuité
    \begin{frame}{Continuité}
    
        Logique d’amélioration continue
        
    \end{frame}
    
    \begin{frame}{Cohabitation}
    
        Prend en compte les autres possibilités
        
    \end{frame}


\section{Faiblesses}

    % Pas de marketing
    \begin{frame}{Communication}
    
        Perception déformée de ce que sont les logiciels libres
        
    \end{frame}
    
    % Configuration manuelle souvent nécessaire
    \begin{frame}{Configuration}
    
        La plupart du temps, une configuration manuelle est nécessaire lors de l’installation.
        
    \end{frame}
    
    % Peu de solutions logicielles éditées pour les distro GNU/Linux
    \begin{frame}{Applications}
    
        Peu d’applications éditées aussi pour les systèmes GNU/Linux
        
    \end{frame}
    
    % 


\section{Opportunités}

    % stagnation du marché des ordis 
    \begin{frame}{Moins d’achat d’ordinateurs}
    
        Les logiciels libres sont plus efficaces en consommation de ressources
        
    \end{frame}
    
    % Défiance vis à vis des USA
    \begin{frame}{Politique mondiale}
    
        Trump est-il un homme de confiance ? \pause Faut-il être si facile à pirater par les Russes ?
        
    \end{frame}
    
    % Hardware open-source
    \begin{frame}{Matériel Open-Source}
    
        ex: RISC-V et autres processeurs utilisables sans binary blobs.
        
    \end{frame}
    
    % Politique intérieure
    \begin{frame}{Politique  intérieure}
    
        Politique d’encouragement des logiciels libres
        
    \end{frame}
    
    % Open Data
    \begin{frame}{Open-data}
    
        Pas mal de données disponibles publiquement exploitables (ex plugin Vélib pour Gnome)
        
    \end{frame}
    
    % Radio cause commune
    \begin{frame}{Radio Cause Commune}
    
        Dédiée aux communs en général, 93,1 à Paris
        
    \end{frame}
    
\section{Menaces}

    % smartphones / tablettes
    \begin{frame}{Smartphones et tablettes}
    
        L’usage est de plus en plus d’utiliser des smartphones ou une tablette.
        
    \end{frame}
    
    % Accords commerciaux
    \begin{frame}{Partenariats}
    
        Microsoft avec l’Éducation Nationale ou la Défense \pause
        mais aussi avec de nombreuses jeunes entreprises.
        
    \end{frame}
    
    % VR
    \begin{frame}{Réalité virtuelle}
    
        Accaparée par GAFAM
        
    \end{frame}
    
    % AI
    \begin{frame}{Intelligences artificielles}
    
        Pas vraiment d’assistant vocal personnel codé en LL
        
    \end{frame}
    
    % Communauté LL
    \begin{frame}{Communauté}
    
        Le renouvellement de la communauté de développeurs est-il suffisant ?
        
    \end{frame}
    
    % Diffusion
    \begin{frame}{Canaux de distribution}
    
        Diffusion via les canaux standards
        
    \end{frame}
    


\section{Conclusions}

    \begin{frame}{Le marché existe}
    
        Dans la façon d’imposer les nouvelles technologies, il y aura toujours des abandonnés.
        
    \end{frame}
    
    \begin{frame}{Est-il mature ?}
    
        Peu de gens comprennent ce qu’est un logiciel ou un système d’exploitation, surtout les débutants.
        
    \end{frame}
    
    \begin{frame}{Opportunité ou menace ?}
        
        A mon sens, c’est équilibré. \pause C’est jouable.
        
    \end{frame}
    
    
\section{Suites}

    \begin{frame}{Liste d’idées}
    
        \begin{enumerate}
            \item Intégrer des services publics en tant que WebApp par défaut
            \item Découpler et unifier la gestion des contacts des applis
            \item Améliorer encore davantage la qualité des logiciels (bugs/perfs/graphisme)
            \item Standardisation des interfaces, notamment menus encore plus poussée. Etendre aux sites webs.
            
        \end{enumerate}
        
    \end{frame}

    \begin{frame}{Participer à cette présentation}
    
        Sur GitHub : \url{
        
        En direct : j’attends vos remarques :)
        
    \end{frame}

    
    
\end{document}

%idées :
% - scripts d'installation et de configuration par modèle
% - liste en dur de sites institutionnels / services publics en lien avec la localisation du système.
% - découpler la gestion des contacts
% - check list "le logiciel est-il bon pour vous ?" 